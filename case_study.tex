\chapter{Case study}
\label{chp:case_study} 

In order to extract the requirements our solution needs, we use the methodology developed by S-Cube network of intelligence.
CITATION NEEDED.

\section{Domain Assumptions}\label{sec:domain_assumptions}


\begin{center}
    \begin{tabular}{ | l | l | l | p{5cm} |}
    \hline
    Day & Min Temp & Max Temp & Summary \\ \hline
    Monday & 11C & 22C & A clear day with lots of sunshine.  
    However, the strong breeze will bring down the temperatures. \\ \hline
    Tuesday & 9C & 19C & Cloudy with rain, across many northern regions. Clear spells
    across most of Scotland and Northern Ireland,
    but rain reaching the far northwest. \\ \hline
    Wednesday & 10C & 21C & Rain will still linger for the morning.
    Conditions will improve by early afternoon and continue
    throughout the evening. \\
    \hline
    \end{tabular}
\end{center}

\begin{center}
	\caption{CAPTION HER}
	\begin{tabular}{ | l | l |}
	\hline
	Field & Description
	
\end{center}

\subsection{Source code example}

% \floatname{algorithm}{Source code} % if you want to rename 'Algorithm' to 'Source code'
\begin{algorithm}[h]
  \caption{The Hello World! program in Java.}
  \label{hello_world}
  % alternatively you may use algorithmic, or lstlisting from the listings package
  \begin{verbatim}
  
class HelloWorldApp {
  public static void main(String[] args) {
    //Display the string
    System.out.println("Hello World!");
  }
}
\end{verbatim}
\end{algorithm}

You can refer to figures using the predefined command like \fref{fig:example}, to pages like \pref{fig:example}, to tables like \tref{tab:example}, to chapters like \Cref{chp:example} and to sections like \Sref{sec:first_section} and you may define similar commands to refer to proposition, algorithms etc.
