\chapter{Applications Today}
\label{chp:Applications Today} 

%Here is an example of how to use acronyms such as \gls{ntnu}. The second time only \gls{ntnu} is shown and if there were several you would write \glspl{ntnu}. And here is an example\footnote{A footnote} of citation~\cite{Author:year:XYZ}. 

%\Blindtext[3][1]

%\begin{figure}
%\centering
%% dummy figure replacement 
%\begin{tabular}{@{}c@{}}
%\rule{.5\textwidth}{.5\textwidth} \\
%\end{tabular}
%\caption{\label{fig:example}A figure}
%\end{figure}

\section{Cross-device Applications}\label{sec:Cross-device Applications}

   When it comes to designing a cross-device application or service, there are things we need to take into consideration. Making an application “cross-device” is not something mandatory in order to make it more efficient or attractive to the user. Even though more and more applications cross the border of being exclusive for a platform, device or OS, we have to ask the question: “How will this change affect the overall experience of the user?” before making a specific application available to more devices and platforms.
\\   So in order to answer the question of whether or not it is worth it, we have to consider the type of the application and the context it is used in. For example, the reason for which more and more “web” applications appear to become available for mobiles, tablets and desktops, is that the quality of the user experience is not degraded in the transition from one platform to another. Web-pages are designed in such a way (Responsive Web Design) so that a web-page is rendered differently depending on the user's browsing device. 
\\ Generally speaking, the so important user-experience is depending on two properties of the user interface: First and foremost, the factor of the appearance and usability. For example, if a user finds it difficult to read the news from a website on his mobile phone/tablet – due to the size of the screen for example – , he will choose to do it from a laptop or desktop instead. Secondly, the input required for an application. In the case of a web-browsing application, the input required is simply clicking on a link or filling out a form, which can be done efficiently from both stationary and mobile devices. In the case of a gaming application, the input required is very different from one device to another, so making the application cross-device wouldn't really serve the purpose of making it more available, since the users are very unlikely to use another device which offers them a worse overall experience.
%\subsection{First subsection with some \texorpdfstring{$\mathcal{M}ath$}{Math} symbol}\label{sec:first_ssection}

%\blindtext

\section{Application Categories}
\subsection{Categorization Criteria}


\subsection{Source code example}



You can refer to figures using the predefined command like \fref{fig:example}, to pages like \pref{fig:example}, to tables like \tref{tab:example}, to chapters like \Cref{chp:example} and to sections like \Sref{sec:first_section} and you may define similar commands to refer to proposition, algorithms etc.
